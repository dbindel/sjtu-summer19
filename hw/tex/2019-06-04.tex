\documentclass[12pt, leqno]{article} %% use to set typesize
\input{common}

\begin{document}
\hdr{2019-06-04}{2019-06-11}

\paragraph*{1: Preserving positive definiteness}
Let $k(x,y)$ be a positive definite kernel function on some set
$\Omega$.  Suppose $g : \Omega' \rightarrow \Omega$ is a one-one
map and $h : \Omega' \rightarrow \bbR$ is nonzero for every $u \in \Omega'$.
Argue that
\[
  \hat{k}(u,v) \equiv k(g(u), g(v)) h(u) h(v)
\]
is a positive definite kernel function on $\Omega'$.  Why are the
hypotheses that $g$ is one-one and that $h$ is nonzero needed?

\paragraph*{2: Sample smoothness}
A standard method for sampling from a multivariate normal distribution
$\mathcal{N}(\mu, K)$ is to compute a Cholesky factorization
$K = R^T R$ and then sample
\[
  Y = R^T Z + \mu
\]
where $Z \mathcal{N}(0, I)$, i.e.~$Z$ is a vector whose entries are
independent standard normal random variables.  Use this technique to
plot draws from a mean zero GP on $[-1,1]$ using the exponential
kernel and the squared exponential kernel with length scales
$\ell = 0.1, 0.5, 1, 2$.  Comment on the apparent smoothness of the samples.

\end{document}
