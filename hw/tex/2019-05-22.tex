\documentclass[12pt, leqno]{article} %% use to set typesize
\input{common}

\begin{document}
\hdr{2019-05-22}{2019-05-22}

\paragraph*{1: Constrained least squares}
Consider the least squares problem
\[
  \mbox{minimize } \|Ax-b\|^2 \mbox{ s.t. } \sum_{j=1}^n x_i = 1.
\]
Write down the KKT conditions for this problem and write a short code
in your favorite language to solve the problem given an economy QR
factorization $A = QR$.

\paragraph*{2: Residual sensitivity}
In this exercise, we will analyze how sensitive the norm of the least
squares residual is to perturbations in $A$.
\begin{enumerate}
\item
  Differentiate the expression $r^T r = \|r\|^2$ to show that
  \[
    \delta \|r\| = \frac{r^T (\delta r)}{\|r\|}.
  \]
\item
  Differentiate the relationship $A^T r = 0$ and pre-multiply by $r^T$
  to get that
  \[
    \delta \|r\| = \frac{-r^T (\delta A) x}{\|r\|}.
  \]
  {\em Hint:} It helps to recall that $A^T r = 0$.
\end{enumerate}
From the last inequality, a standard norm bound shows that
\[
  \delta \|r\| \leq \|\delta A\| \|x\|,
\]
i.e.~the residual magnitude can only change significantly with a small
change to $A$ if the coefficients in the least squares solve were large.


\end{document}
