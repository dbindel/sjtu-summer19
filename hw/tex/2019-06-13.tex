\documentclass[12pt, leqno]{article} %% use to set typesize
\input{common}

\begin{document}
\hdr{2019-06-13}{2019-06-24}

\paragraph*{1: Ring around the rosie}
Consider the undirected ring on $n$ nodes.  For $n = 5$, the Laplacian
is
\[
L = \begin{bmatrix}
       2 & -1 &  0  &  0 & -1 \\
      -1 &  2 & -1  &  0 &  0 \\
       0 & -1 &  2  & -1 &  0 \\
       0 &  0 & -1  &  2 & -1 \\
      -1 &  0 &  0  & -1 &  2
    \end{bmatrix},
\]
and similarly for larger $n$.  For $n = 10$, plot the graph in 2D by
placing each node according to the Laplacian eigenmap coordinates.
Repeat with a tiny random change to the edge weights; what happens to
the coordinates, and why?
If you are using MATLAB, you may want to use the {\tt gplot} command.

\paragraph*{2: Almost reducible chains}
Consider an almost irreducible Markov chain with transition matrix
\[
  P = P^{\mathrm{ref}} + E, \quad
  P^{\mathrm{ref}} = \begin{bmatrix} P_{11} & 0 \\ 0 & P_{22} \end{bmatrix}
\]
and suppose $P_{11}$ and $P_{22}$ are both ergodic with unique
stationary vectors $\pi_1^*$ and $\pi_2^*$, respectively.
We can approximate the stationary distribution of $P$ by
\[
  \hat{\pi} = \begin{bmatrix} \alpha \pi_1 \\ (1-\alpha) \pi_2 \end{bmatrix};
\]
show how to choose $\alpha$ to minimize the residual error
$\|(I-P) \hat{\pi}\|^2$.

\end{document}
