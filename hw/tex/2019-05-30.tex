\documentclass[12pt, leqno]{article} %% use to set typesize
\input{common}

\begin{document}
\hdr{2019-05-30}{2019-06-05}

\paragraph*{1: HALS-RRI}
Implement the RRI iteration and apply it to the test problem in
{\tt demo\_nmf}.  Use {\tt svd\_nmf\_init} for an initial starting
point, and run for 100 iterations.  Demonstrate the convergence
by giving a semilog plot of $r_k-r_{\mathrm{final}}$ where $r = \|A-WH\|_F$.

\paragraph*{2: AA-HALS-RRI}
Acceleration methods convert a slowly-converging sequences into more
rapidly convergent sequences by learning patterns in the relations
between steps.  Anderson acceleration is an acceleration method
that applies specifically to fixed point iterations of the form
\[
  x^{k+1} = G(x^k)
\]
transforming to a new iteration
\[
  \tilde{x}^{k+1} = \sum_{j=0}^{m-1} \alpha_j G(\tilde{x}^{k-j})
\]
where the coefficients $\alpha_j$ are learned from looking
at the relation between $\tilde{x}^{k-j}$ and $G(\tilde{x}^{k-j})$
over several steps.
%
A simple code to run a step of Anderson acceleration is included
in the class repository.  Use this code to accelerate the convergence
of RRI in the previous problem, starting about 20 steps in.
Again demonstrate the convergence by giving a semilog plot
of the $r_k-r_{\mathrm{final}}$.

{\bf Note:} In order to use Anderson acceleration, you will need
to pack the matrices $W$ and $H$ into a single vector $x$ at each
step (and then unpack afterward).  You can do this using MATLAB's
{\tt reshape} command.


\end{document}
