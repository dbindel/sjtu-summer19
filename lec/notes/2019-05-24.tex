\documentclass[12pt, leqno]{article} %% use to set typesize
\input{common}

\begin{document}
\hdr{2019-05-24}

\section{Stochastic gradient methods}

In the last half of the last lecture, we discussed the gradient descent
iteration
\[
  x^{k+1} = x^k - \alpha_k \nabla \phi(x^k).
\]
For small enough fixed $\alpha$ and nice enough $\phi$, we can guarantee
that the error scales like $\|e^k\| = O(\rho^k)$ for some $\rho < 1$.
This type of convergence is known by optimizers as
{\em (R)-linear convergence}, and in machine learning it is sometimes
called {\em geometric convergence}.  We also saw last time that we can
sometimes still obtain convergence results even when
$\nabla \phi(x^k)$ is not computed exactly, as long as the errors in the
gradient computation are controlled in some way.

The {\em stochastic gradient} methods replace $\nabla \phi(x)$ by a
randomized estimator.  These methods are typically applied to
objective functions that consist of a large number of independent
terms, e.g.
\[
  \phi(x) = \frac{1}{N} \sum_{i=1}^N \phi_i(x)
\]  
In this case, we can {\em randomly sample} the $\phi_i$ in order to
obtain an unbiased estimate of the gradient, e.g.
\[
  \nabla \phi(x) = \mathbb{E}_i [\nabla \phi_i(x)].
\]
This estimator is unbiased, but the variance is high; in order to
reduce the variance, one sometimes uses randomly-selected
``minibatches'' of points
\[
  \nabla \phi(x) =
  \mathbb{E}_{\mathcal I} \left[
    \frac{1}{|\mathcal{I}|} \sum_{i \in \mathcal{I}} \nabla \phi_i(x)
  \right].
\]
Let's call such estimators $g(x, \xi)$, where $\xi$ is a random
variable that determines the selection of data used in the estimator.
Then the stochastic gradient algorithm is
\[
  x^{k+1} = x^k - \alpha_k g(x^k, \xi_k).
\]

What does the convergence of the stochastic gradient algorithm
look like?  For nice enough functions and a sufficiently small
fixed step size $\alpha$, the expected values optimality gap behaves like
\[
  \mathbb{E}[\phi(x^k)-\phi(x^*)] \leq
  c_1 \alpha + (1-c_2 \alpha)^{-k} \left( \phi(x^0)-\phi(x^*) \right)
\]
That is, the expected optimality gap converges linearly, but not to
zero!  To get closer to the true optimal value, we have to reduce the
step size.  Unfortunately, reducing the step size also reduces the
rate of convergence!  We can balance the two effects by taking $n_0$
steps with an initial size of $\alpha_0$ to get the error down to
$O(\alpha_0)$, $2 n_0$ steps of size $2^{-1} \alpha_0$ to get the
error down to $O(2^{-1} \alpha)$, and so forth.  This gives us a
convergence rate of $O(1/k)$.  More generally, we can get convergence
with any (sufficiently small) schedule of step sizes such that
\[
  \sum_{k=1}^\infty \alpha_k = \infty, \quad
  \sum_{k=1}^\infty \alpha_k^2 < \infty.
\]
There are a wide variety of methods for choosing the step sizes
(``learning rate''), sometimes in conjunction with methods to choose
a better search direction than the (approximate) steepest descent direction.

The $O(1/k)$ rate of convergence for stochastic gradient descent is quite slow
compared to the rate of convergence for ordinary gradient descent.
However, each step of a stochastic gradient method may be much
cheaper, so there is a tradeoff.  The slow rate of the stochastic
gradient method comes from a combination of two effects: variance in
the gradient estimates, and the slow rate of gradient descent when the
problem is ill-conditioned.

\section{Scaling Steepest Descent}

Let us put aside, for now, the stochastic methods and instead
return to gradient descent.  We saw last time that with an optimal
step size, the convergence of gradient descent on a positive definite
quadratic model problem behaves like
\[
  \|e^k\| \leq \rho^k \|e^0\|, \quad \mbox{ where }
  \rho = 1 - O(\kappa(A)^{-1}),
\]
where $\kappa(A) = \lambda_{\max}(A)/\lambda_{\min}(A)$ is the
condition number of $A$.  Hence, if $\kappa(A)$ is large (the problem
is {\em ill-conditioned}), then convergence can be quite slow.
Sometimes slow convergence is a blessing in disguise, as we saw last
time, but sometimes we really do want a faster method.  What can we
do?

A natural generalization of steepest descent is {\em scaled} steepest
descent.  In this iteration, we choose a positive definite matrix $M$,
and use the iteration
\begin{align*}
  p^{k}   &= -M^{-1} \nabla \phi(x^k) \\
  x^{k+1} &= x^k + \alpha_k p^k.
\end{align*}
The {\em search direction} $p^k$ is no longer the direction of steepest
descent, but it {\em is} still a descent direction; that is, if
$\nabla \phi(x^k) \neq 0$, then for small enough $\epsilon$,
\[
  \phi(x^k + \epsilon p^k) =
  \phi(x^k) + \epsilon \nabla \phi(x^k)^T p^k + O(\epsilon^2) < \phi(x^k)
\]
since
\[
  \nabla \phi(x^k)^T p^k = -\nabla \phi(x^k)^T M^{-1} \nabla
  \phi(x^k) < 0
\]
by positive definiteness of $M^{-1}$.

The convergence for our quadratic model function
\[
  \phi(x) = \frac{1}{2} x^T A x + b^T x + c
\]
is determined by the error iteration
\[
  e^{k+1} = (I-\alpha_k M^{-1} A) e^k.
\]
In this case, the optimal choice of $M$ and $\alpha$ would be
$M = A$ and $\alpha = 1$; in this case, the iteration converges
in a single step!  Of course, it is too much to ask for convergence
in one step when our objective is more complicated.  Still, the
quadratic model tells us a lot.  If $x^*$
is a local strong minimum and $\phi$ is sufficiently smooth,
we have the local Taylor approximation
\[
  \phi(x^* + z) = \phi(x^*) + \frac{1}{2} z^T H_{\phi}(x^*) z + O(\|z\|^3)
\]
where $H_{\phi}(x^*)$ is the Hessian matrix
\[
  \left[ H_{\phi}(x^*) \right]_{ij} =
  \frac{\partial^2 \phi(x^*)}{\partial x_i \partial x_j}.
\]
For initial points $x^0$ near enough to $x^*$, we have
\[
  e^{k+1} = e^k - H_\phi(x^*)^{-1}
  \left[ \nabla \phi(x^* + e^k) - \nabla \phi(x^*) \right],
\]
and substituting
\[
  \nabla \phi(x^* + e^k) - \nabla \phi(x^*) =
  H_{\phi}(x^*) e^k + O(\|e^k\|^2),
\]
we have
\[
  \|e^{k+1}\|
  = \|e^k - H_{\phi}(x^*)^{-1} H_{\phi}(x^*) e^k\| + O(\|e^k\|^2)
  = O(\|e^k\|^2).
\]
This is known as {\em quadratic} convergence.

\section{Newton's Method and Line Search}

One problem with using $H_{\phi}(x^*)$ as a scaling matrix is that we
don't know where $x^*$ is --- if we did, we would have no need for an
optimization algorithm!  However, we can approximate this optimal
scaling.  One natural choice is to scale with $H_{\phi}(x^k)$;
this gives us {\em Newton's method}.  This method is equivalent
to at every step solving the quadratic optimization problem
\[
p^k = \operatorname{argmin}_p
\phi(x^k) + \nabla \phi(x^k)^T p + \frac{1}{2} p^T H_\phi(x^k) p.
\]
Newton's method has the disadvantage that we have a new scaling matrix
at every step (and so may need to do a new factorization), but it has
the advantage that the approximation to $H_{\phi}(x^*)$ gets better
and better as $x^k \rightarrow x^*$.  Indeed, this method is also
locally quadratically convergent.

Newton's method converges very quickly once we are close to a strong
minimizer $x^*$ such that the Hessian is positive definite.  But far
away from $x^*$, the problem may run into trouble in two ways: we
could have an indefinite Hessian matrix, so that the Newton direction
is not a descent direction; or a full Newton step might go too far,
causing the iteration to diverge.  To deal with the issue of
indefiniteness, we use an alternate scaling matrix when indefiniteness
is detected.  For example, we might use $H_{\phi}(x^k) + \lambda_k I$,
which can always be made positive definite with a sufficiently
positive choice of $\lambda_k$.  To deal with the issue of not taking
too long a step, we use a {\em globalization} technique; the two
common approaches are {\em trust regions} or {\em line search}.  For
simplicity, we will focus on the latter.

For step size choices for Newton and related iterations, we want
to satisfy two conditions.  First, the step sizes should not go to
zero, or at least they should not go to zero so quickly that the
iteration can misconverge.  Second, there should be ``sufficient
decrease'' at each step, i.e.~we want to satisfy the condition
\[
  \phi(x^k + \alpha_k p^k) \leq \phi(x^k) + c \alpha_k \nabla
  \phi(x^k)^T p^k,
\]
for some $c < 1$.  We typically choose $c$ quite small, so usually
this condition is equivalent to just making sure that $\phi(x^{k+1})$
is less than $\phi(x^k)$.  The simplest method to choose the step
size to satisfy these conditions is a {\em backtracking line search}:
we start with a step size of one, then repeatedly cut the step in half
until the sufficient decrease condition holds.



\section{Gauss-Newton}

We turn now to another popular iterative solver: the Gauss-Newton
method for nonlinear least squares problems.  Given
$f : \bbR^n \rightarrow \bbR^m$ for $m > n$, we seek to minimize
the objective function
\[
  \phi(x) = \frac{1}{2} \|f(x)\|^2.
\]
The Gauss-Newton approach to this optimization is to approximate
$f$ by a first order Taylor expansion in order to obtain a proposed
step:
\[
  p_k
    = \operatorname{argmin}_p \frac{1}{2} \|f(x_k) + f'(x_k) p\|^2
    = -f'(x_k)^\dagger f(x_k).
\]
Writing out the pseudo-inverse more explicitly, we have
\begin{align*}
  p_k
  &= -[f'(x_k)^T f'(x_k)]^{-1} f'(x_k)^T f(x_k) \\
  &= -[f'(x_k)^T f'(x_k)]^{-1} \nabla \phi(x_k).
\end{align*}
The matrix $f'(x_k)^T f'(x_k)$ is positive definite if $f'(x_k)$ is
full rank; hence, the direction $p_k$ is always a descent direction
provided $x_k$ is not a stationary point and $f'(x_k)$ is full rank.
However, the Gauss-Newton step is {\em not} the same as the Newton
step, since the Hessian of $\phi$ is
\[
  H_{\phi}(x) = f'(x)^T f'(x) + \sum_{j=1}^m f_j(x) H_{f_j}(x).
\]
Thus, the Gauss-Newton iteration can be seen as a modified Newton
in which we drop the inconvenient terms associated with second
derivatives of the residual functions $f_j$.

Assuming $f'$ is Lipschitz with constant $L$, an error analysis about a
minimizer $x_*$ yields
\[
  \|e_{k+1}\| ~ \leq ~ L \|f'(x_*)^\dagger\|^2 \|f(x_*)\| \|e_k\| + O(\|e_k\|^2).
\]
Thus, if the optimal residual norm $\|f(x_*)\|$ is small, then from
good initial guesses, Gauss-Newton converges nearly quadratically
(though the linear term will eventually dominate).  On the other had,
if $\|f(x_*)\|$ is larger than $\|f'(x_*)^\dagger\|$, then the
iteration may not even be locally convergent unless we apply some type
of globalization strategy.

\section{Regularization and Levenberg-Marquardt}

While we can certainly apply line search methods to globalize
Gauss-Newton iteration, an alternate proposal due to Levenberg and
Marquardt is to solve a {\em regularized} least squares problem to
compute the step; that is,
\[
p_k =
\operatorname{argmin}_p
  \frac{1}{2} \|f(x_k) + f'(x_k) p\|^2 +
  \frac{\lambda}{2} \|Dp\|^2.
\]
The scaling matrix $D$ may be an identity matrix (per Levenberg),
or we may choose $D^2 = \operatorname{diag}(f'(x_k)^T f'(x_k))$
(as suggested by Marquardt).

For $\lambda = 0$, the Levenberg-Marquardt step is the same as
a Gauss-Newton step.  As $\lambda$ becomes large, though, we have
the (scaled) gradient step
\[
  p_k = -\frac{1}{\lambda} D^{-2} f(x_k) + O(\lambda^{-2}).
\]
Unlike Gauss-Newton with line search, changing the parameter $\lambda$
affects not only the distance we move, but also the direction.

In order to get both ensure global convergence (under sufficient
hypotheses on $f$, as usual) and to ensure that convergence is not
too slow, a variety of methods have been proposed that adjust
$\lambda$ dynamically.  To judge whether $\lambda$ has been chosen
too aggressively or conservatively, we monitor the {\em gain ratio}, or the
ratio of actual reduction in the objective to the reduction predicted
by the (Gauss-Newton) model:
\[
  \rho =
  \frac{\|f(x_k)\|^2-\|f(x_k+p_k)\|^2}
       {\|f(x_k)\|^2 - \|f(x_k)+f'(x_k)p_k\|^2}.
\]
If the step decreases the function value enough ($\rho$ is
sufficiently positive), then we accept the step; otherwise,
we reject it.  For the next step (or the next attempt), we
may increase or decrease the damping parameter $\lambda$
depending on whether $\rho$ is close to one or far from one.

\end{document}
